\documentclass{article}
\usepackage[spanish]{babel}
\usepackage[utf8]{inputenc}
\usepackage{amsmath}
\usepackage{amssymb}
\usepackage{cancel}

\usepackage{amsthm}
\newtheorem{teo}{Teorema}

\usepackage{anysize}
\marginsize{1.3cm}{1.3cm}{1.3cm}{1.3cm}

\usepackage[usenames]{color}
\definecolor{azul}{RGB}{10,80,190}

\title{
    Matemáticas para las Ciencias Aplicadas IV\\
    Tarea-Examen 03 
}
\author{
    Careaga Carrillo Juan Manuel \\
    Quiróz Castañeda Edgar \\
    Soto Corderi Sandra del Mar
}
\date{
    03 de mayo de 2019
}
\begin{document}
    \maketitle
    {\bf Resuelve las siguientes ecuaciones diferenciales}
    \begin{enumerate}
        
        % Ejercicio 1
        \item {
            $(t - 2)^2 \ddot y + 5(t-2)\dot y+4y=0$

            \color{azul}
            % Respuesta
            
        }
        % Ejercicio 2
        \item {
            $t^2 \ddot y+ 3t \dot y + 2y =0$

            \color{azul}
            % Respuesta
            
        }
        % Ejercicio 3
        \item {
            Usar el método de reducción de orden para demostrar que 
            $y_2(t) = t ^ {r_1} ln t$ cuando se tienen raíces repetidas en la 
            ecuación de Euler. 

            \color{azul}
            % Respuesta
           
        }
        % Ejercicio 4
        \item {
            $2t^2\ddot y + 3t\dot y - (1 - t)y = 0$

            \color{azul}
            % Respuesta

        }

        % Ejercicio 5
        \item {
            $2t^2\ddot y + (t^2 - t)\dot y + y = 0$

            \color{azul}
            La ecuación se puede escribir en su forma estándar dividiendo por $2t^2$
            \[
                \ddot y + \frac{t^2-t}{2t^2}\dot y + \frac{1}{2t^2}y = 0
            \]
            Tanto $P(t)=\frac{t^2-t}{2t^2}$ como $Q(t)=\frac{1}{2t^2}$ se indeterminan cuando $t=0$, por lo que $t_0=0$
            es un punto singular, veamos que $t_0$ es regular pues tanto $t\cdot P(t)$ como $t^2\cdot Q(t)$ son
            analíticas.
            \[
                \lim_{t_0 \to 0}{t\cdot\frac{t^2-t}{2t^2}}
                =\lim_{t_0\to 0}{\frac{t-1}{2}}
                =-\frac{1}{2}=p_0
            \]
            \[
                \lim_{t_0\to 0}{t^2\cdot\frac{1}{2t^2}}
                =\lim_{t_0\to 0}{\frac{1}{2}}
                =\frac{1}{2}=q_0
            \]
            La ecuación indicial asociada al a ecuación diferencial es $r^2+(-\frac{1}{2}-1)r+\frac{1}{2}=0$, que
            simplificada es $2r^2-3r+1=0$ y cuyas raíces son $r_1=1$ y $r_2=\frac{1}{2}$

            Como $r_1=r_2+\lambda$, siendo $\lambda$ un entero fraccionario, se propone que ambas soluciones sean de la
            forma
            $$y(t)=t^r\sum_{n=0}^{\infty}{c_nt^n}=\sum_{n=0}^{\infty}{c_nt^{n+r}}$$
            por lo que
            $$\dot y(t)=\sum_{n=0}^{\infty}{c_n(n+r)t^{n+r-1}}$$
            $$\ddot y(t)=\sum_{n=0}^{\infty}{c_n(n+r)(n+r-1)t^{n+r-2}}$$
            Sustituyendo en la ecuación original, tenemos
            \[
                2t^2\sum_{n=0}^{\infty}{c_n(n+r)(n+r-1)t^{n+r-2}}
                +(t^2-t)\sum_{n=0}^{\infty}{c_n(n+r)t^{n+r-1}}
                +\sum_{n=0}^{\infty}{c_nt^{n+r}} = 0
            \]
            \[
                \sum_{n=0}^{\infty}{2c_n(n+r)(n+r-1)t^{n+r}}
                +\sum_{n=0}^{\infty}{c_n(n+r)t^{n+r+1}}
                -\sum_{n=0}^{\infty}{c_n(n+r)t^{n+r}}
                +\sum_{n=0}^{\infty}{c_nt^{n+r}}=0
            \]
            \[
                \sum_{n=0}^{\infty}{[2(n+r)(n+r-1)-(n+r)+1]c_nt^{n+r}}
                +\sum_{n=0}^{\infty}{c_n(n+r)t^{n+r+1}}=0
            \]
            Calculamos el primer término de la primera suma para tener el mismo exponente en la $t$ después de ajustar
            los índices
            \[
                \left(2r(r-1)-r+1\right)c_0t^r
                +\sum_{n=1}^{\infty}{[2(n+r)(n+r-1)-(n+r)+1]c_nt^{n+r}}
                +\sum_{n=0}^{\infty}{c_n(n+r)t^{n+r+1}}=0
            \]
            \[
                \left(2r^2-2r-r+1\right)c_0t^r
                +\sum_{n=0}^{\infty}{[2(n+r+1)(n+r)-(n+r+1)+1]c_{n+1}t^{n+r+1}}
                +\sum_{n=0}^{\infty}{c_n(n+r)t^{n+r+1}}=0
            \]
            \[
                \left(2r^2-3r+1\right)c_0t^r
                +\sum_{n=0}^{\infty}{
                    \left\{
                        \left[
                            2(n+r+1)(n+r)-n-r
                        \right]c_{n+1}
                        +
                        (n+r)c_n
                    \right\}
                    t^{n+r+1}
                }=0
            \]
            Igualamos todos los coeficientes a cero. Notemos que el primer término se reduce a la ecuación indicial:\\
            $2r^2-3r+1=0$ cuyas raíces recordemos son $r_1=1$ y $r_2=\frac{1}{2}$. Con los demás coeficientes obtenemos
            la relación de recurrencia
            \[
                [2(n+r+1)-1](n+r)c_{n+1}+(n+r)c_n=0
            \]
            \[
                (2n+2r+1)(n+r)c_{n+1}=-(n+r)c_n
            \]
            \[
                c_{n+1}=-\frac{\cancel{(n+r)}c_n}{(2n+2r+1)\cancel{(n+r)}}
                =-\frac{c_n}{2n+2r+1}
            \]
            Para $r=r_1=1$ tenemos que $c_{n+1}=-\frac{c_n}{2n+3}$
            \begin{center}
                \begin{tabular}{ll}
                    si $n=0$ & $c_1=-\frac{c_0}{3}$ \\
                    si $n=1$ & $c_2=-\frac{c_1}{5}=\frac{c_0}{15}$ \\
                    si $n=2$ & $c_3=-\frac{c_2}{7}=-\frac{c_0}{105}$ \\
                    si $n=3$ & $c_4=-\frac{c_3}{9}=\frac{c_0}{945}$ \\
                    si $n=4$ & $c_5=-\frac{c_4}{11}=-\frac{c_0}{10395}$ \\
                    $\vdots$ & $\vdots$
                \end{tabular}
            \end{center}
            por lo que
            \[
                y_1(t)=c_0t\left[
                    1-\frac{t}{3}+\frac{t^2}{15}-\frac{t^3}{105}+\frac{t^4}{945}-\ldots
                \right]
            \]
            Para $r=r_2=\frac{1}{2}$ tenemos que $c_{n+1}=-\frac{c_n}{2(n+1)}$
            \begin{center}
                \begin{tabular}{ll}
                    si $n=0$ & $c_1=-\frac{c_0}{2}$ \\
                    si $n=1$ & $c_2=-\frac{c_1}{4}=\frac{c_0}{8}$ \\
                    si $n=2$ & $c_3=-\frac{c_2}{6}=-\frac{c_0}{48}$ \\
                    si $n=3$ & $c_4=-\frac{c_3}{8}=\frac{c_0}{384}$ \\
                    si $n=4$ & $c_5=-\frac{c_4}{10}=-\frac{c_0}{3840}$ \\
                    $\vdots$ & $\vdots$
                \end{tabular}
            \end{center}
            por lo que
            \[
                y_2(t)=c_0t^{\frac{1}{2}}\left[
                    1-\frac{t}{2}+\frac{t^2}{8}-\frac{t^3}{48}+\frac{t^4}{384}-\ldots
                \right]
            \]
            Entonces, la solución general de la ecuación diferencial es
            \[
                y(t)=c_0\left[
                    t\left(1-\frac{t}{3}+\frac{t^2}{15}-\frac{t^3}{105}+\frac{t^4}{945}-\ldots\right)
                    +t^{\frac{1}{2}}\left(1-\frac{t}{2}+\frac{t^2}{8}-\frac{t^3}{48}+\frac{t^4}{384}-\ldots\right)
                \right]
            \]
        }

        % Ejercicio 6
        \item {
            Invertigar las soluciones del método de Frobenius cuando $r_2 - r_1$
            es entero y cuando $r_1 = r_2$.

            \color{azul}
            \begin{teo}
                \footnote{\c{C}engel, Y.\&Palm W.. (2014). Ecuaciones diferenciales para ingeniería y ciencias. México:
                McGraw Hill.}
                Sea el punto $x=0$ un punto singular regular de la ecuación diferencial:
                $$y''+P(x)y'+Q(x)y=0$$
                y sea $\rho$ el menor radio de convergencia de dos funciones $p(x)=xP(x)$ y $q(x)=x^2Q(x)$. Si $r_1$ y
                $r_2$ son las raíces de la ecuación indicial:
                $$r^2+(p_0-1)r+q_0=0$$
                donde
                \begin{eqnarray*}
                    p_0=\lim_{x\to 0}xP(x) & \text{ y } & q_0=\lim_{x\to 0}x^2Q(x)
                \end{eqnarray*}
                y $r_1>r_2$ cuando las raíces son reales y desiguales, entonces existen dos soluciones linealmente
                independientes $y_1(x)$ y $y_2(x)$ de esta ecuación diferencial, con un radio de convergencia de
                $\rho$. Para $x>0$, son de las siguientes formas:

                {\bf Caso 1:} $r_1=r_2+\lambda$ ($\lambda$ es positiva no entera)
                \begin{eqnarray*}
                    y_1=x^{r_1}\sum_{n=0}^{\infty}{a_nx^n} & (a_0\neq 0) \\
                    y_2=x^{r_2}\sum_{n=0}^{\infty}{b_nx^n} & (b_0\neq 0) 
                \end{eqnarray*}
                {\bf Caso 2:} $r_1=r_2=r$
                \begin{eqnarray*}
                    y_1=x^{r}\sum_{n=0}^{\infty}{a_nx^n} & (a_0\neq 0) \\
                    y_2=y_1\ln{x}+x^{r}\sum_{n=1}^{\infty}{b_nx^n}
                \end{eqnarray*}
                {\bf Caso 3:} $r_1=r_2+N$ ($N$ es un entero positivo)
                \begin{eqnarray*}
                    y_1=x^{r_1}\sum_{n=0}^{\infty}{a_nx^n} & (a_0\neq 0) \\
                    y_2=Cy_1\ln{x}+x^{r_2}\sum_{n=0}^{\infty}{b_nx^n} & (b_0\neq 0)
                \end{eqnarray*}
                donde la constante $C$ puede ser cero. Entonces, la solución general de la ecuación diferencial para
                los tres casos se expresa como
                $$y=C_1y_1(x)+C_2y_2(x)$$
                donde las constantes arbitrarias $C_1$ y $C_2$ se determinan por las condiciones iniciales y en la
                frontera.
            \end{teo}
        }

        % Ejercicio 7
        \item {
            $t\ddot y - (4 + t)\dot y + 2y = 0$

            \color{azul}
            Obtenemos la forma estándar al dividir por $t$
            $$\ddot y-\frac{t+4}{t}\dot y+\frac{2}{t}y=0$$
            $t_0=0$ es un punto singular regular, pues
            \begin{eqnarray*}
                \lim_{t_0\to 0}{-\frac{t+4}{t}t}=\lim_{t_0\to 0}{-t-4}=-4=p_0 &
                \lim_{t_0\to 0}{\frac{2}{t}t^2}=\lim_{t_0\to 0}{2t}=0=q_0
            \end{eqnarray*}
            De ahí, la ecuación indicial queda $r^2-5r=0$ y sus raíces son $r_1=5$ y $r_2=0$.

            Por el teorema anterior, la ecuación diferencial dada tiene dos soluciones linealmente independientes: la
            segunda puede contener un término logarítmico.

            Proponemos como la primera solución
            $$y_1(t)=\sum_{n=0}^{\infty}{a_nt^{n+r_1}}$$
            Entonces
            $$\dot y_1(t)=\sum_{n=0}^{\infty}{a_n(n+r_1)t^{n+r_1-1}}$$
            $$\ddot y_1(t)=\sum_{n=0}^{\infty}{a_n(n+r_1)(n+r_1-1)t^{n+r_1-2}}$$
            Sustituyendo
            \[
                t\sum_{n=0}^{\infty}{a_n(n+r_1)(n+r_1-1)t^{n+r_1-2}}
                -(4+t)\sum_{n=0}^{\infty}{a_n(n+r_1)t^{n+r_1-1}}
                +2\sum_{n=0}^{\infty}{a_nt^{n+r_1}}=0
            \]
            \[
                \sum_{n=0}^{\infty}{a_n(n+r_1)(n+r_1-1)t^{n+r_1-1}}
                -\sum_{n=0}^{\infty}{4a_n(n+r_1)t^{n+r_1-1}}
                -\sum_{n=0}^{\infty}{a_n(n+r_1)t^{n+r_1}}
                +2\sum_{n=0}^{\infty}{a_nt^{n+r_1}}=0
            \]
            \[
                \sum_{n=0}^{\infty}{a_n(n+r_1)t^{n+r_1-1}[n+r_1-1-4]}
                +\sum_{n=0}^{\infty}{a_nt^{n+r_1}[2-n-r_1]}=0
            \]
            \[
                a_0r_1t^{r_1-1}(r_1-5)
                +\sum_{n=1}^{\infty}{a_n(n+r_1)(n+r_1-5)t^{n+r_1-1}}
                +\sum_{n=0}^{\infty}{a_n(2-n-r_1)t^{n+r_1}}=0
            \]
            \[
                a_0r_1(r_1-5)t^{r_1-1}
                +\sum_{n=0}^{\infty}{a_{n+1}(n+r_1+1)(n+r_1-4)t^{n+r_1}}
                +\sum_{n=0}^{\infty}{a_n(2-n-r_1)t^{n+r_1}}=0
            \]
            \[
                a_0r_1(r_1-5)t^{r_1-1}
                +\sum_{n=0}^{\infty}{[
                    a_{n+1}(n+r_1+1)(n+r_1-4)
                    +a_n(2-n-r_1)
                ]t^{n+r_1}}=0
            \]
            Igualamos los términos a cero, del primer término obtenemos la ecuación indicial, y del resto tendremos
            \[a_{n+1}(n+r_1+1)(n+r_1-4)+a_n(2-n-r_1)=0\]
            \[a_{n+1}(n+r_1+1)(n+r_1-4)=a_n(n+r_1-2)\]
            \begin{equation}
                a_{n+1}=\frac{a_n(n+r_1-2)}{(n+r_1+1)(n+r_1-4)}
                \label{recurrente}
            \end{equation}
            Como $r_1=5$
            \[a_{n+1}=\frac{a_n(n+3)}{(n+6)(n+1)}\]
            \begin{center}
                \begin{tabular}{ll}
                    si $n=0$ & $a_1=\frac{3a_0}{6}=\frac{a_0}{2}$ \\
                    si $n=1$ & $a_2=\frac{4a_1}{(7)(2)}=\frac{a_0}{7}$ \\
                    si $n=2$ & $a_3=\frac{5a_2}{(8)(3)}=\frac{5a_0}{168}$ \\
                    si $n=3$ & $a_4=\frac{6a_3}{(9)(4)}=\frac{5a_0}{1008}$ \\
                    si $n=4$ & $a_5=\frac{7a_4}{(10)(5)}=\frac{a_0}{1440}$ \\
                    \vdots   & \vdots
                \end{tabular}
            \end{center}
            Por lo que 
            \[
                y_1(t)=a_0t^5\left[
                    1+\frac{t}{2}+\frac{t^2}{7}+\frac{5t^3}{168}+\frac{5t^4}{1008}+\ldots
                \right]
            \]
            Para la segunda solución linealmente independiente proponemos
            $$y_2(t)=Cy_1\ln{t}+t^{r_2}\sum_{n=0}^{\infty}{b_nt^{n}}$$
            Es decir
            $$y_2(t)=Cy_1\ln{t}+t^{0}\sum_{n=0}^{\infty}{b_nt^{n}}$$
            $$y_2(t)=Cy_1\ln{t}+\sum_{n=0}^{\infty}{b_nt^{n}}$$
            Supongamos que ésta solución carece de término logaritmico, es decir, $C=0$, por lo que la propuesta se
            simplificará a
            $$y_2(t)=\sum_{n=0}^{\infty}{b_nt^{n}}$$
            Como es parecida a $y_1(t)$, el procedimiento para encontrar los coeficientes es idéntico hasta la ecuación
            (\ref{recurrente}), por lo que
            \[
                b_{n+1}=\frac{b_n(n-2)}{(n+1)(n-4)}
            \]
            \begin{center}
                \begin{tabular}{ll}
                    si $n=0$ & $b_1=\frac{-2b_0}{-4}=\frac{b_0}{2}$ \\
                    si $n=1$ & $b_2=\frac{-b_1}{(2)(-3)}=\frac{b_0}{12}$ \\
                    si $n=2$ & $b_3=\frac{0b_2}{(3)(-2)}=0$ \\
                \end{tabular}
            \end{center}
            En este punto nos detenemos, pues al haber un cero el resto de coeficientes serán cero.

            Por lo tanto
            \[
                y_2(t)=b_0\left(1+\frac{t}{2}+\frac{t^2}{12}\right)
            \]
            Que se puede comprobar fácilmente que corresponde a una solución de la ecuación diferencial y además es
            linealmente independiente a $y_1(t)$.

            Por lo tanto la solución buscada es
            \[
                y(t)=
                a_0t^5\left[1+\frac{t}{2}+\frac{t^2}{7}+\frac{5t^3}{168}+\frac{5t^4}{1008}+\ldots\right]
                +b_0\left(1+\frac{t}{2}+\frac{t^2}{12}\right)
            \]
        }

        % Ejercicio 8 
        \item {
            $\ddot y - 5 \dot y + 4y = e^{2t}$ con $y(0) = 1$ y $\dot y(0) = -1$.

            \color{azul}
            % Respuesta
        }

        % Ejercicio 9
        \item {
            $\ddot y + y = t\sen t$ con $y(0) = 1$ y  $\dot y(0) = 2$.

            \color{azul}
            % Respuesta
        }

        % Ejercicio 10
        \item {
            $\ddot y + \dot y + y = 1 + e^{-t}$ con $y(0) = 3$ y $\dot y(0) = -5$.

            \color{azul}
            % Respuesta
        }
    \end{enumerate}
\end{document}
\documentclass{article}
\usepackage[spanish]{babel}
\usepackage[utf8]{inputenc}
\usepackage{amsmath}
\usepackage{amssymb}

\usepackage{anysize}
\marginsize{1.3cm}{1.3cm}{1.3cm}{1.3cm}

\usepackage[usenames]{color}
\definecolor{azul}{RGB}{10,80,190}

\title{
    Matemáticas para las Ciencias Aplicadas IV\\
    Tarea-Examen 03 
}
\author{
    Careaga Carrillo Juan Manuel \\
    Quiróz Castañeda Edgar \\
    Soto Corderi Sandra del Mar
}
\date{
    03 de mayo de 2019
}
\begin{document}
    \maketitle
    {\bf Resuelve las siguientes ecuaciones diferenciales}
    \begin{enumerate}
        
        % Ejercicio 1
        \item {
            $(t - 2)^2 \ddot y + 5(t-2)\dot y+4y=0$ con $t > 0$ \\

            \color{azul}
            % Respuesta
            
            Utilizamos el método de la Ecuación de Euler para encontrar la solución general de la ecuación, podemos utilizar este método ya que la ecuación es de la forma $(t - t_0)^2\ddot{y} + \alpha (t - t_0) \dot{y} + \beta y = 0$ con $\alpha$ y $\beta$ constantes. Esta ecuación es también una ecuación de Euler, con una singularidad en $t = t_0$ en lugar de $t = 0$. En este caso buscamos por soluciones de la forma $(t - t_0)^r$\\
            
            Queremos ver en que caso cae de acuerdo a las raíces de su ecuación característica, así que substituimos $y = (t-2)^r$ en la ecuación y obtenemos las raíces de la siguiente forma:
            
            $r(r -1)(t -2)^r + 5r(t-2)^r + 4(t-2)^r$\\
            $[r(r-1) + 5r + 4] (t-2)^r$\\
            $[r^2 + 4r + 4] (t-2)^r$\\
            $(r+2)^2(t-2)^r$\\
            
            La ecuación $(r+2)^2 = 0$ tiene $r = -2$ como raíz repetida, así que caemos en el Caso 2.\\
            De ahí, $y_1(t-2) = (t-2)^{-2}, \ \  y_2(t-2) = (t-2)^{-2}ln(t-2)$\\
            
             Por lo tanto, tenemos que la solución general de la ecuación es:\\
            
            $y(t) = (c_1 + c_2 ln(t-2))(t-2)^{-2} $ \\
            
            $y(t) = \frac{c_1 + c_2 ln(t-2)}{(t-2)^2} $\\
            
            
            
        }
        % Ejercicio 2
        \item {
            $t^2 \ddot y+ 3t \dot y + 2y =0$ con $t > 0$ \\

            \color{azul}
            % Respuesta
             Utilizamos el método de la Ecuación de Euler para encontrar la solución general de la ecuación, podemos utilizar este método ya que la ecuación es de la forma $t^2\ddot{y} + \alpha t \dot{y} + \beta y = 0$ con $\alpha$ y $\beta$ constantes.\\
            
            Queremos ver en que caso cae de acuerdo a las raíces de su ecuación característica, así que substituimos $y = t^r$ en la ecuación y obtenemos las raíces de la siguiente forma:
            
            $r(r -1)t^r + 3rt^r + 2t^r$\\
            $[r(r-1) + 3r + 2] t^r$\\
            $[r^2 + 2r + 2] t^r$\\
            Las raíces de la ecuación $r^2 + 2r + 2 = 0$ son:\\
            $\frac{-2 \pm \sqrt{4 - 8}}{2} = -1\pm i$\\
            
           Tenemos la siguiente solución compleja:\\
            
            \begin{align*}
             \phi(t) &= t^{-1 + i} = t^{-1}t^{i}\\
             &= t^{-1}e^{(lnt)i} = t^{-1}e^{i(lnt)}\\
             &= t^{-1}[cos(lnt) + isen(lnt)]
            \end{align*}
        
        	Consecuentemente vemos que al tener soluciones complejas, caemos en el Caso 3 y tenemos:
        	
            $y_1(t) = Re\{\phi(t)\} = t^{-1}cos(lnt) $ y 
            $y_2(t) = Im\{\phi(t)\} = t^{-1}sen(lnt) $\\
            
          
           Por lo tanto, la solución general de la ecuación es:\\
            
            $y(t) = (c_1cos(ln(t)) + c_2 sen(ln(t)))(t)^{-1} $ \\
            
            $y(t) = \frac{c_1cos(ln(t)) + c_2 sen(ln(t))}{t} $\\
            
            
        }
        % Ejercicio 3
        \item {
            Usar el método de reducción de orden para demostrar que 
            $y_2(t) = t ^ {r_1} ln t$ cuando se tienen raíces repetidas en la 
            ecuación de Euler. 

            \color{azul}
            % Respuesta
            El método de reducción de orden nos permite usar una solución conocida $Y_1$ de una ecuación diferencial lineal de segundo orden homogénea para encontrar una solución lineal independiente de $Y_2$\\
            
            Así que consideramos la ecuación de Euler $t^2\ddot{y} + \alpha t \dot{y} + \beta y = 0$ y suponemos que $y_1$, donde $y_1 = t^{r_1}$ es una solución a la ecuación. Y $r_1 = \frac{1 - \alpha}{2}$ \\
            
            Dividimos la ecuación de Euler entre $t^2$
            $\ddot{y} + \frac{\alpha}{t}\dot{y} + \frac{\beta}{t^2}y = 0$\\
            
            Sabemos que para resolver una ecuación diferencial de segundo orden, debemos tener dos soluciones linealmente independientes. Así que, debemos determinar una segunda solución linealmente independiente y lo obtenemos al intentar encontrar una solución de la forma: $y_2 = v(t) y_1 = v(t) t^{r_1}$ donde $v(t)$ no es una función constante, ya que si fuera constante, $y_1$ y $y_2$ serían linealmente dependientes.\\
                        
            Calculamos $\dot{y_2}(t) = \dot{y_1} v(t) + \dot{v}(t)y_1  $\\
            Calculamos $\ddot{y_2}(t) = \ddot{y_1} v(t) + \dot{v}(t)\dot{y_1} + \dot{y_1}\dot{v}(t) + \ddot{v}(t)y_1  $\\
            
            Sustituimos en la ecuación dividida de Euler:
            
            $ [\ddot{y_1} v(t) + 2\dot{v}(t)\dot{y_1} + \ddot{v}(t)y_1] + \frac{\alpha}{t} [\dot{y_1} v(t) + \dot{v}(t)y_1] + \frac{\beta}{t^2} [v(t) y_1] = 0$\\
            
            Multiplicamos por $t^2$ y factorizamos:
            
            $[t^2y_1]\ddot{v}(t) + [t^2(2\dot{y_1} + \frac{\alpha}{t}y_1)]\dot{v}(t) + [t^2\ddot{y_1} + \alpha t \dot{y_1} + \beta y_1]v(t) = 0$\\
            
            Sabemos que la ecuación de Euler es igual a cero, así que eliminamos el factor multiplicado por $v(t)$ y volvemos a dividir entre $t^2$\\
            $[y_1]\ddot{v}(t) + [2\dot{y_1} + \frac{\alpha}{t}y_1]\dot{v}(t) = 0$\\
            
            Hacemos un cambio de variable donde $w = \dot{v}(t) \ y \  \dot{w} = \ddot{v}(t)$\\
             $[y_1]\dot{w} + [2\dot{y_1} + \frac{\alpha}{t}y_1]w = 0$\\
             
             Ahora, queremos obtener w, así que derivamos $\dot{w}$\\
             $\frac{dw}{dt}[y_1] + w[2\dot{y_1} + \frac{\alpha}{t}y_1] = 0$\\
             $\frac{dw}{dt}[y_1] = - w[2\dot{y_1} + \frac{\alpha}{t}y_1]$\\
             $\frac{dw}{dt} = - w[2\frac{\dot{y_1}}{y_1} + \frac{\alpha}{t}]$\\
             $\frac{dw}{w} = - [2\frac{\dot{y_1}}{y_1}dt + \frac{\alpha}{t}dt]$\\
             
             Integramos e ignoramos las constantes que se acumulen por el momento:\\
             $\int \frac{dw}{w} dt + 2 \int \frac{\dot{y_1}}{y_1} dt + \int \frac{\alpha}{t} dt = 0 $\\
             
             Recordemos que $y_1$ es una función de t\\
             $ln(w) + 2ln(y_1) + \alpha ln(t) = 0$\\
             
             Usamos leyes de los logaritmos\\
             $ln(w) + ln(y_1^2) + \alpha ln(t) = 0$\\
             $ln(w y_1^2) = -\alpha ln(t) $\\
             
             Aplicamos epsilon de los dos lados\\
             $wy_1^2 = e^{-\alpha ln(t)}$\\
             $w = \frac{e^{-\alpha ln(t)}}{y_1^2}$\\
             
             Regresamos a las variables originales y tenemos:\\
             $\dot{v} = w = \frac{e^{-\alpha ln(t)}}{y_1^2}$\\
             
             Integramos de nuevo para obtener v e ignoramos las constantes por el momento\\
             $\int \dot{v}(t) dt = \int \frac{e^{-\alpha ln(t)}}{y_1^2} dt$\\
             $v(t) = \int \frac{t^{-\alpha}}{y_1^2}dt$\\
             $v(t) = \int \frac{t^{-\alpha}}{y_1^2}dt$\\
             
             Sustituimos el valor de $y_1$ en la integral\\
             $v(t) = \int \frac{t^{-\alpha}}{(t^{r_1})^2}dt$\\
             $v(t) = \int \frac{1}{t^{2r_1 + \alpha}}dt$\\
             
             Sustituimos el valor de $r_1$ en la integral\\
             $v(t) = \int \frac{1}{t^{2(\frac{1 - \alpha}{2}) + \alpha}}dt$\\
             $v(t) = \int \frac{1}{t^{1 - \alpha + \alpha}}dt$\\
             $v(t) = \int \frac{1}{t^{1}}dt$\\
             
             Finalmente obtenemos el resultado de la integral\\
              $v(t) = ln(t)$\\
              
             Por lo tanto $y_2 = t^{r_1}ln(t)$ $\blacksquare$\\
             
            
            
            
            
        }
        % Ejercicio 4
        \item {
            $2t^2\ddot y + 3t\dot y - (1 + t)y = 0$

            \color{azul}
            % Respuesta
			Queremos utilizar el método de Frobenius, así que verificamos que $x= 0$ sea un punto singular para $ t^2\ddot{y} + \frac{3}{2}t\dot{y} - \frac{(1 + t)}{2} y$\\
			
			Veamos si los límites existen:
			
			$\lim_{t \rightarrow 0} \frac{\frac{3}{2} t}{t^2} (t-0) = \frac{\frac{3}{2}t^2}{t^2} = \frac{3}{2}$\\
			$\lim_{t \rightarrow 0} \frac{-(1+t)}{2t^2} (t-0)^2 = \frac{-(1+t)t^2}{t^2} = \frac{-(1+t)}{2} = -\frac{1}{2}$\\
			
			Como $x = 0$ es un punto singular, podemos aplicar el método de Frobenius.
			
			Proponemos $\sum_{n = 0}^{\infty} a_n t ^{n + r}, \ a_0 \neq 0$\\
			Calculamos $\dot{y}(t) = \sum_{n = 0}^{\infty} (n + r)a_n t ^{n + r -1}$\\
			Y $\ddot{y}(t) = \sum_{n = 0}^{\infty} (n + r)(n + r -1)a_n t ^{n + r -2}$\\
			
			Y sustituimos en la ecuación:\\
			$ \sum_{n = 0}^{\infty} (n + r)(n + r -1)a_n t ^{n + r} + \frac{3}{2}\sum_{n = 0}^{\infty} (n + r)a_n t ^{n + r} - \frac{(1 + t)}{2} \sum_{n = 0}^{\infty} a_n t ^{n + r}$\\
			$ \sum_{n = 0}^{\infty} [(n + r)(n + r -1) + \frac{3}{2}(n + r)  - \frac{1 }{2}] a_n t ^{n + r}  + \sum_{n = 0}^{\infty} - \frac{1}{2} a_{n} t ^{n + r+ 1}$\\
			
			Proponer la suma de coeficientes como potencias de t igual a cero da:
			
			$(r)(r -1) + \frac{3}{2}(r)  - \frac{1}{2}$\\
			$r^2 + \frac{1}{2}r - \frac{1}{2}$\\
			$(r + 1)(r - \frac{1}{2})$\\
			
			Entonces tenemos nuestras raíces $r_1 = -1, r_2 = \frac{1}{2}$, como estas raíces no difieren por un entero, podemos encontrar dos soluciones independientes de la forma $\sum_{n = 0}^{\infty} a_n t ^{n + r}$ con $a_n$ determinada por  
			
			$[(n + r)(n + r -1) + \frac{3}{2}(n + r)  - \frac{1 }{2}]a_n = \frac{1}{2}a_{n-1}$
			
			
			Por lo tanto la solución general de la ecuación es:\\
			
			$\frac{c_1}{t}(1 - \frac{t^2}{2!} - \frac{t^3}{3 \cdot 3 !} - \frac{t^4}{3 \cdot 5 \cdot 4 !} - \dots) + c_2 \sqrt{t} (1 + \frac{t}{5} + \frac{t^2}{5 \cdot 7 \cdot 2 !} + \frac{t^3}{5 \cdot 7 \cdot 9 \cdot 3 !} + \dots)$
        }

        % Ejercicio 5
        \item {
            $2t^2\ddot y + (t^2 - t)\dot y + y = 0$

            \color{azul}
            % Respuesta

        }

        % Ejercicio 6
        \item {
            Invertigar las soluciones del método de Frobenius cuando $r_2 - r_1$
            es entero y cuando $r_1 = r_2$.

            \color{azul}
            % Respuesta
        }

        % Ejercicio 7
        \item {
            $t\ddot y - (4 + t)\dot y + 2y = 0$

            \color{azul}
            % Respuesta
        }

        % Ejercicio 8 
        \item {
            $\ddot y - 5 \dot y + 4y = e^{2t}$ con $y(0) = 1$ y $\dot y(0) = -1$.

            \color{azul}
            % Respuesta
        }

        % Ejercicio 9
        \item {
            $\ddot y + y = t\sen t$ con $y(0) = 1$ y  $\dot y(0) = 2$.

            \color{azul}
            % Respuesta
        }

        % Ejercicio 10
        \item {
            $\ddot y + \dot y + y = 1 + e^{-t}$ con $y(0) = 3$ y $\dot y(0) = -5$.

            \color{azul}
            % Respuesta
        }
    \end{enumerate}
\end{document}
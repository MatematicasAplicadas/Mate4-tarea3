\documentclass{article}
\usepackage[spanish]{babel}
\usepackage[utf8]{inputenc}
\usepackage{amsmath}
\usepackage{amssymb}

\usepackage{anysize}
\marginsize{1.3cm}{1.3cm}{1.3cm}{1.3cm}

\usepackage[usenames]{color}
\definecolor{azul}{RGB}{10,80,190}

\title{
    Matemáticas para las Ciencias Aplicadas IV\\
    Tarea-Examen 03 
}
\author{
    Careaga Carrillo Juan Manuel \\
    Quiróz Castañeda Edgar \\
    Soto Corderi Sandra del Mar
}
\date{
    03 de mayo de 2019
}
\begin{document}
    \maketitle
    {\bf Resuelve las siguientes ecuaciones diferenciales}
    \begin{enumerate}
        
        % Ejercicio 1
        \item {
            $(t - 2)^2 \ddot y + 5(t-2)\dot y+4y=0$\\

            \color{azul}
            % Respuesta
            
        }
        % Ejercicio 2
        \item {
            $t^2 \ddot y+ 3t \dot y + 2y =0$\\

            \color{azul}
            % Respuesta
            
        }
        % Ejercicio 3
        \item {
            Usar el método de reducción de orden para demostrar que 
            $y_2(t) = t ^ {r_1} ln t$ cuando se tienen raíces repetidas en la 
            ecuación de Euler. 

            \color{azul}
            % Respuesta
           
        }
        % Ejercicio 4
        \item {
            $2t^2\ddot y + 3t\dot y - (1 - t)y = 0$

            \color{azul}
            % Respuesta

        }

        % Ejercicio 5
        \item {
            $2t^2\ddot y + (t^2 - t)\dot y + y = 0$

            \color{azul}
            % Respuesta

        }

        % Ejercicio 6
        \item {
            Invertigar las soluciones del método de Frobenius cuando $r_2 - r_1$
            es entero y cuando $r_1 = r_2$.

            \color{azul}
            % Respuesta
        }

        % Ejercicio 7
        \item {
            $t\ddot y - (4 + t)\dot y + 2y = 0$

            \color{azul}
            % Respuesta
        }

        % Ejercicio 8 
        \item {
            $\ddot y - 5 \dot y + 4y = e^{2t}$ con $y(0) = 1$ y $\dot y(0) = -1$.

            \color{azul}
            % Respuesta
        }

        % Ejercicio 9
        \item {
            $\ddot y + y = t\sen t$ con $y(0) = 1$ y  $\dot y(0) = 2$.

            \color{azul}
            % Respuesta
        }

        % Ejercicio 10
        \item {
            $\ddot y + \dot y + y = 1 + e^{-t}$ con $y(0) = 3$ y $\dot y(0) = -5$.

            \color{azul}
            % Respuesta
        }
    \end{enumerate}
\end{document}
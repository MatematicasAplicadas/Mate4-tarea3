\documentclass{article}
\usepackage[spanish]{babel}
\usepackage[utf8]{inputenc}
\usepackage{amsmath}
\usepackage{amssymb}

\usepackage{anysize}
\marginsize{1.3cm}{1.3cm}{1.3cm}{1.3cm}

\usepackage[usenames]{color}
\definecolor{azul}{RGB}{10,80,190}

\title{
    Matemáticas para las Ciencias Aplicadas IV\\
    Tarea-Examen 03 
}
\author{
    Careaga Carrillo Juan Manuel \\
    Quiróz Castañeda Edgar \\
    Soto Corderi Sandra del Mar
}
\date{
    03 de mayo de 2019
}
\begin{document}
    \maketitle
    {\bf Resuelve las siguientes ecuaciones diferenciales}
    \begin{enumerate}
        
        % Ejercicio 1
        \item {
            $(t - 2)^2 \ddot y + 5(t-2)\dot y+4y=0$\\

            \color{azul}
            % Respuesta
            
        }
        % Ejercicio 2
        \item {
            $t^2 \ddot y+ 3t \dot y + 2y =0$\\

            \color{azul}
            % Respuesta
            
        }
        % Ejercicio 3
        \item {
            Usar el método de reducción de orden para demostrar que 
            $y_2(t) = t ^ {r_1} ln t$ cuando se tienen raíces repetidas en la 
            ecuación de Euler. 

            \color{azul}
            % Respuesta
           
        }
        % Ejercicio 4
        \item {
            $2t^2\ddot y + 3t\dot y - (1 - t)y = 0$

            \color{azul}
            % Respuesta

        }

        % Ejercicio 5
        \item {
            $2t^2\ddot y + (t^2 - t)\dot y + y = 0$

            \color{azul}
            % Respuesta

        }

        % Ejercicio 6
        \item {
            Invertigar las soluciones del método de Frobenius cuando $r_2 - r_1$
            es entero y cuando $r_1 = r_2$.

            \color{azul}
            % Respuesta
        }

        % Ejercicio 7
        \item {
            $t\ddot y - (4 + t)\dot y + 2y = 0$

            \color{azul}
            % Respuesta
        }

        % Ejercicio 8 
        \item {
            $\ddot y - 5 \dot y + 4y = e^{2t}$ con $y(0) = 1$ y $\dot y(0) = -1$.

            \color{azul}
            % Respuesta
            Primero, hay que sacar la transformada de Laplace de toda la ecuación
            para despejar $Y(s)$.
            \begin{align*}
                \mathcal{L}\{\ddot y - 5 \dot y + 4y\} &= \mathcal{L}\{e^{2t}\} \\
                \mathcal{L}\{\ddot y\} - 5\mathcal{L}\{\dot y\} + 4 \mathcal{L}\{y\}
                &= \mathcal{L}\{e^{2t}\} \\
                s^2 Y(s) - s y(0) - \dot y(0) - 5 (sY(s) - y(0)) + 4Y(s) 
                &= \frac{1}{s - 2} \\
                s^2 Y(s) - s + 1 - 5s Y(s) + 5 + 4 Y(s) &= \frac{1}{s - 2} \\
                Y(s) (s^2 - 5s + 4) + s + 6 &= \frac{1}{s - 2} \\
                Y(s) &= \frac{1}{(s-2)(s-4)(s-1)} - \frac{(s+6)}{(s-4)(s-1)}\\
            \end{align*}

            Luego, descomponiendo en fracciones parciales.\\
            Primero
            \begin{align*}
                \frac{1}{(s-2)(s-4)(s-1)}=&\frac{A}{s-2} + \frac{B}{s-4} + \frac{C}{s-1} \\
                =&\frac{A(s-4)(s-1) + B(s-2)(s-1) + C(s-2)(s-4)}{(s-2)(s-4)(s-1)} \\
                &\iff A(s-4)(s-1) + B(s-2)(s-1) + C(s-2)(s-4) = 1
            \end{align*}
            Evaluando en $s = 2$, tenemos
            \[-4A = 1 \iff A = -\frac{1}{4}\]
            Evaluando en $s = 4$ tenemos
            \[6B = 1 \iff B = \frac{1}{6}\]
            Evaluando en $s = 1$ tenemos
            \[3C = 1 \iff C = \frac{1}{3}\]

            Segundo
            \begin{align*}
                \frac{s+6}{(s-4)(s-1)} &= \frac{D}{s-4} + \frac{E}{s-1} \\
                &= \frac{D(s-1) + E(s-4)}{(s-4)(s-1)} \\
                &\iff D(s-1) + E(s-4) = s+6
            \end{align*}
            Evaluando en $s = 4$ tenemos que
            \[3D = 10 \iff D = \frac{10}{3}\]
            Evaluando en $s = 1$ tenemos que
            \[-3E = 7 \iff E = -\frac{7}{3}\]

            Juntando estas dos cosas, tenemos que 
            \begin{align*}
                Y(s) &= -\frac{1}{4} \frac{1}{s-2} + \frac{1}{6} \frac{1}{s-4}
                \frac{1}{3} \frac{1}{s-1} + \frac{10}{3} \frac{1}{s-4} 
                - \frac{7}{3} \frac{1}{s-1} \\
                &= -\frac{1}{4} \frac{1}{s-2} + \frac{7}{2}\frac{1}{s-4} 
                - 2\frac{1}{s-1}
            \end{align*}
            De aquí tenemos que 
            \begin{align*}
                \mathcal{L}^{-1}\{Y(s)\} &= \mathcal{L}^{-1}\{\frac{1}{4} \frac{1}{s-2} 
                + \frac{7}{2}\frac{1}{s-4} - 2\frac{1}{s-1}\} \\
                y(t) &= \frac{1}{4}\mathcal{L}^{-1}\{\frac{1}{s-2}\}
                +\frac{7}{2}\mathcal{L}^{-1}\{\frac{1}{s-4}\}
                -2\mathcal{L}^{-1}\{\frac{1}{s-1}\} \\
                y(t) &= \frac{1}{4}e^{2t}+\frac{7}{2}e^{4t}-2e^{t}
            \end{align*}
            Que es la solución buscada.
        }

        % Ejercicio 9
        \item {
            $\ddot y + y = t\sen t$ con $y(0) = 1$ y  $\dot y(0) = 2$.

            \color{azul}
            % Respuesta
            Primero, hay que sacar la transformada de Laplace de toda la
            ecuación para despejar $Y(s)$.
            \begin{align*}
                \mathcal{L}\{\ddot y + y\} &= \mathcal{L}\{t\sen t\} \\
                \mathcal{L}\{\ddot y\} + \mathcal{L}\{y\}
                &= -\frac{d}{ds}\mathcal{L}\{\sen t\}\\
                s^2 Y(s) - s y(0) - \dot y(0) + Y(s) 
                &= -\frac{d}{ds}\frac{1}{s^2+1}\\
                s^2 Y(s) - s - 2 + Y(s) &= \frac{2s}{(s^2+1)^2} \\
                Y(s) (s^2+1) - s - 2 &= \frac{2s}{(s^2+1)^2}\\
                Y(s) &= \frac{2s}{(s^2+1)^3} + \frac{s+2}{s^2+1} \\
                Y(s) &= 2\frac{s}{s^2+1}\frac{1}{s^2+1}\frac{1}{s^2+1}
                + \frac{s}{s^2+1} + 2\frac{1}{s^2+1}
            \end{align*}
            Ahora, hay que sacar la transformada inversa.
            \begin{align*}
                \mathcal{L}^{-1}\{Y(s)\} 
                    &= \mathcal{L}^{-1}\{
                        2\frac{s}{s^2+1}\frac{1}{s^2+1}\frac{1}{s^2+1}
                        + \frac{s}{s^2+1} + 2\frac{1}{s^2+1}
                    \} \\
                y(t) &= 2\mathcal{L}^{-1}\{\frac{s}{s^2+1}\frac{1}{s^2+1}\frac{1}{s^2+1}\} 
                    + \mathcal{L}^{-1}\{\frac{s}{s^2+1}\} + 2\mathcal{L}^{-1}\{\frac{1}{s^2+1}\}
                    \\
                y(t) &= 2\cos{t}\sin^{2}{t} + \cos{t} + 2\sin{t}
            \end{align*}
            Que es la solución buscada.
        }

        % Ejercicio 10
        \item {
            $\ddot y + \dot y + y = 1 + e^{-t}$ con $y(0) = 3$ y $\dot y(0) = -5$.

            \color{azul}
            % Respuesta
        
            Primero hay qu sacar la transformada de 
            Laplace de toda la ecuacioń, e intentar despejar $Y(s)$.
            \begin{align*}
                \mathcal{L}\{\ddot y + \dot y + y\} &= \mathcal{L}\{1 + e^{-t}\}\\
                \mathcal{L}\{\ddot y\} + \mathcal{L}\{\dot y\} + \mathcal{L}\{y\}
                &= \mathcal{L}\{1\} + \mathcal{L}\{e^{-t}\} \\
                s^2 Y(s) - s y(0) - \dot y(0) + sY(s) - y(0) + Y(s) &= \frac{1}{s}
                + \frac{1}{s+1} \\
                s^2 Y(s) - 3s + 5 + s Y(s) - 3 + Y(s) &= \frac{1}{s} + \frac{1}{s+1}\\
                Y(s) (s^2 + s + 1) - (3s-2) &= \frac{1}{s} + \frac{1}{s+1}\\
                Y(s) &= \frac{1}{s(s^2 + s + 1)} + \frac{1}{(s+1)(s^2 + s + 1)} 
                + \frac{3s-2}{s^2 + s + 1}
            \end{align*}
            Luego, descompongamos algunas de las expresiones en fraccciones
            parciales.
            Primero
            \begin{align*}
                \frac{1}{s(s^2+s+1)} &= \frac{A}{s} + \frac{Bs + C}{s^2+s+1} \\
                &= \frac{As^2+As+A+Bs^2+Cs}{s(s^2+s+1)} \\
                &\iff 1 = As^2+As+A+Bs^2+Cs
            \end{align*}
            Que nos define un sistemas de ecuaciones lineales.

            Para resolverlo, primero obtenemos
            \[A = 1\]
            Y usando este valor de $A$, podemos obtener
            \[A + C = 0 \implies C = -1\]
            Y
            \[A + B = 0 \implies B = -1\]
            Luego, para la otra expresión
            \begin{align*}
                \frac{1}{(s+1)(s^2 + s + 1)} &= \frac{A}{s+1} + \frac{Bs+C}{s^2+s+1} \\
                &= \frac{As^2+As+A+Bs^2+Bs+Cs+C}{s(s^2+s+1)} \\
                &\iff 1 = 
            \end{align*}
            Que defines un sistema de ecuaciones.
            \begin{align*}
                A + B &= 0 \\
                A + B + C &= 0 \\
                A + C &= 1
            \end{align*}
            Para resolverlo, primero restamos las dos primeras ecuaciones
            \[(A+B)-(A+B+C) = 0 - 0 \implies C = 0\]
            Sustituyendo en la tercera ecuación, obtenemos
            \[A + 0 = 1 \implies A = 1\]
            Y sustituyendo esto en la primera tenemos que
            \[1 + B = 0 \implies B = -1\]

            Con esto, se puede reescribir $Y(s)$ como 
            \begin{align*}
                Y(s) &= \frac{1}{s} - \frac{s+1}{s^2+s+1} + \frac{1}{s+1} 
                - \frac{s}{s^2+s+1} + \frac{3s-2}{s^2+s+1} \\
                Y(s) &= \frac{1}{s} + \frac{1}{s+1} + \frac{s-3}{s^2+s+1} \\
            \end{align*}

            Luego, notemos que $\mathcal{L}\{e^{at}f(t)\} = F(s-a)$, 
            donde $F(s) = \mathcal{L}\{f(t)\}$.

            Por lo que en particular
            \[\mathcal{L}\{e^{-at}\cos(bt)\} = \frac{s+a}{(s+a)^2+b^2}\]
            Y 
            \[\mathcal{L}\{e^{-at}\sin(bt)\} = \frac{b}{(s+a)^2+b^2}\]


            Ahora, vamos a llevar al último término a una forma similar a esta.

            Primera, completamos el cuadrado del denominador
            \[
                \frac{s-3}{s^2+s+1} 
                = \frac{s-3}{s^2+s+\frac{1}{4} + \frac{3}{4}}
                = \frac{s-3}{(s+\frac{1}{2})^2+\frac{3}{4}}
            \]

            Luego, forzamos a que la constante de nominador coincida con la
            constante del cuadrado, aunque esto divide la expresión en dos.
            \[
                \frac{s-3}{(s+\frac{1}{2})^2+\frac{3}{4}} 
                = \frac{s+\frac{1}{2}}{(s+\frac{1}{2})^2+\frac{3}{4}}
                - \frac{7}{2}\frac{1}{(s+\frac{1}{2})^2+\frac{3}{4}}
            \]
            Ahora, en esta otra parte, hay que forzar a que el nominador sea la
            raíz del nominador.
            \[
                \frac{7}{2}\frac{1}{(s+\frac{1}{2})^2+\frac{3}{4}} 
                = \frac{7}{2} \frac{1}{(s+\frac{1}{2})^2+\frac{3}{4}}
                \cdot \frac{\frac{\sqrt{3}}{2}}{\frac{\sqrt{3}}{2}}
                = \frac{7}{\sqrt{3}}\frac{\frac{\sqrt{3}}{2}}{(s+\frac{1}{2})
                +\frac{3}{4}}
            \]
            
            Entonces, se reescribe $Y(s)$ como 
            \[
                Y(s) = \frac{1}{s} + \frac{1}{s+1}
                + \frac{s+\frac{1}{2}}{(s+\frac{1}{2})^2+\frac{3}{4}}
                -\frac{7}{\sqrt{3}}\frac{\frac{\sqrt{3}}{2}}{(s+\frac{1}{2})+\frac{3}{4}}
            \]

            Ahora, resolviendo para encontrar la inversa de esto
            \begin{align*}
                \mathcal{L}^{-1}\{Y(s)\} 
                &= \mathcal{L}^{-1}\{
                    \frac{1}{s} + \frac{1}{s+1}
                    + \frac{s+\frac{1}{2}}{(s+\frac{1}{2})^2+\frac{3}{4}}
                    - \frac{7}{\sqrt{3}}\frac{\frac{\sqrt{3}}{2}}{(s+\frac{1}{2})+\frac{3}{4}}
                \} \\
                y(t) &= \mathcal{L}^{-1}\{\frac{1}{s}\} 
                    + \mathcal{L}^{-1}\{\frac{1}{s+1}\}
                    + \mathcal{L}^{-1}\{\frac{s+\frac{1}{2}}{(s+\frac{1}{2})^2+\frac{3}{4}}\} 
                    - \frac{7}{\sqrt{3}} \cdot \mathcal{L}^{-1}\{\frac{\frac{\sqrt{3}}{2}}{(s+\frac{1}{2})+\frac{3}{4}}\}
                    \\
                y(t) &= 1 + e^{-t} + e^{-\frac{1}{2}t}\sin{\frac{3}{4}t}
                    - \frac{7}{\sqrt{3}}\cdot e^{-\frac{1}{2}t}\cos{\frac{3}{4}t}
            \end{align*}
            Que es la solución buscada.
        }
    \end{enumerate}
\end{document}